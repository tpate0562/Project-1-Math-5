 %!TeX root = Project 1 Math 5.tex
\documentclass{article}
\usepackage[dvipsnames, svgnames, x11names]{xcolor}
\usepackage{tikz}
\usepackage{pgfplots}
\usepackage{setspace}
\usepackage{units}
\usepackage{graphicx}
\usepackage{amsopn}
\usepackage{bbding}
\usepackage{amsmath}
\usepackage{hyperref}
\usepackage{cancel}
\usepackage{gensymb}
\usepackage[margin = 1in]{geometry}
\title{Project 1, Transformations}
\date{25 March, 2025}
\author{Tejas Patel}
\begin{document}
\maketitle
\section{}
\textbf{Matrices defined}: $A=\begin{bmatrix}1&2\\3&4\end{bmatrix},\;\;B=\begin{bmatrix}0&2&-1\\1&1&10\end{bmatrix},\;\; V=\begin{bmatrix}1&4&0\end{bmatrix}$
\\[0.1in]When \verb+show(A), show(B), show(V)+ was input into the python terminal, the output from SageMath was $\begin{bmatrix}1&2\\3&4\end{bmatrix},\;\;\begin{bmatrix}0&2&-1\\1&1&10\end{bmatrix},\;\; \begin{bmatrix}1&4&0\end{bmatrix}$ respectively. The matrices were repeated back to me as I originially entered them
\\[0.01in]$$\text{For }QQ \rightarrow RR \Rightarrow \begin{bmatrix}1.00000000000000&3.00000000000000\\2.00000000000000&4.00000000000000\end{bmatrix}$$ 
$$\text{For }QQ \rightarrow RDF  \Rightarrow \begin{bmatrix}1.0 & 2.0 \\3.0 & 4.0\end{bmatrix}$$ 
\\[0.1in] $RR$ and $RDF$ seem to be data types for the matrices. $RR$ appears to be an aribtrary precision floating point, outputting the maximum number of decimal places allowed. It won't be as precise as $QQ$, which is symbolic math, nor will it be as fast as $RDF$, but it will get you a decimal answer when youre looking for one.
\\[0.1in]$$\text{RREF}(B)=\begin{bmatrix}
    1 & 0 & \frac{21}{2} \\
0 & 1 & -\frac{1}{2}
\end{bmatrix}$$
\\[0.1in] The command creates an identity matrix of the degree of the input. 
\\For the original command: $\begin{bmatrix}1 & 0 & 0 & 0 \\0 & 1 & 0 & 0 \\0 & 0 & 1 & 0 \\0 & 0 & 0 & 1\end{bmatrix}$ 
For input = 2: $\begin{bmatrix}1 & 0 \\0 & 1\end{bmatrix}$
For input = 5: $\begin{bmatrix}1 & 0 & 0 & 0 & 0 \\0 & 1 & 0 & 0 & 0 \\0 & 0 & 1 & 0 & 0 \\0 & 0 & 0 & 1 & 0 \\0 & 0 & 0 & 0 & 1\end{bmatrix}$
\pagebreak
\section{}
\textbf{New matrices defined}: $C=\begin{bmatrix}3 & 7 \\1 & 6\end{bmatrix}$
\\[0.05in]In SageMath, to add: \verb|A+C| results in $\begin{bmatrix}4 & 9 \\4 & 10\end{bmatrix}$, to subtract \verb|A-C| results in $\begin{bmatrix}-2 & -5 \\2 & -2\end{bmatrix}$
\\[0.05in] Lastly, to multiply: \verb|A*C| results in $\begin{bmatrix}5 & 19 \\13 & 45\end{bmatrix}$, bonus: \verb+det(A*C)+ results in -22
$$B^T=\begin{bmatrix}0 & 1 \\2 & 1 \\-1 & 10\end{bmatrix}$$
\\[0.1in] \verb+A*C==C*A+ returned \verb+false+, meaning $AC\neq CA$
\\[0.1in] \verb+(A*C).transpose()==A.transpose()*C.transpose()+ returned \verb+false+, meaning $(AC)^T\neq A^TC^T$
\\[0.1in] \verb|(A+C).transpose()==A.transpose()+C.transpose()| returned \verb+true+, meaning $(A+C)^T=A^T+C^T$
\\[0.1in] \verb|(c*A).transpose()==c*A.transpose()| returned \verb|true| for $c=6$ and $c=10$, meaning $(cA)^T=c*A^T$
\\[0.1in]
\pagebreak
\section{}
\pagebreak
\section{}
\pagebreak
\section{}

\end{document}